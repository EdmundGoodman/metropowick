\documentclass[10pt,aspectratio=169]{beamer}
\usetheme[progressbar=frametitle]{metropolis}
\setbeamertemplate{navigation symbols}{}

\usepackage{appendixnumberbeamer}
\usepackage{metropowick}
\usepackage{booktabs}
\usepackage[scale=2]{ccicons}
\usepackage{subcaption}
\usepackage{tikz}
\usepackage{pgfplots}
\usepgfplotslibrary{dateplot}

\usepackage{minted}
\setminted{
escapeinside=||,%
linenos,%
breaklines,%
autogobble%
}

\usepackage[backend=biber, citestyle=authortitle, maxbibnames=99]{biblatex}
\renewcommand{\cite}{\footcite}
\renewcommand{\footnotesize}{\fontsize{6pt}{6pt}\selectfont}
\addbibresource{references.bib}

\title{Metropowick}\subtitle{A modern beamer theme, for Warwick!}
\date{\today}
\author{Jane Bloggs}
\institute{University of Warwick}

\begin{document}

\maketitle

\begin{frame}{Agenda}
    \setbeamertemplate{section in toc}[sections numbered]
    \tableofcontents
\end{frame}

\section{Elements}

\begin{frame}{Bullets}
    \begin{itemize}
        \item This is a bullet point
        \item This is another bullet point
    \end{itemize}
\end{frame}

\begin{frame}{Columns}
    \begin{columns}[T,onlytextwidth]
        \column{0.5\textwidth}
            \textbf{Items}
            \begin{itemize}
                \item Milk
                \item Eggs
                \item Potatoes
            \end{itemize}

        \column{0.5\textwidth}
            \textbf{Enumerations}
            \begin{enumerate}
                \item First,
                \item Second,
                \item and Last.
            \end{enumerate}
    \end{columns}
\end{frame}

\begin{frame}{Images}
    \begin{figure}[H]
        \includegraphics[width=0.5\textwidth]{example-image-a}
        \caption{A caption \cite{Knuth92}.}
        \label{fig:image}
    \end{figure}
\end{frame}

\begin{frame}[fragile]{Code}
    \begin{listing}[H]
        \begin{minted}[linenos=false]{haskell}
            newtype Lasagne = Lasagne Int
                deriving (Show, Num)

            -- The stacking operating can be considered integer
            -- addition of the number of layers
            instance Semigroup Lasagne where
                (<>) = (+)
        \end{minted}
        \caption{Another caption.}
        \label{lst:listing}
    \end{listing}
\end{frame}

\maketitle

\appendix

\begin{frame}[allowframebreaks]{References}
    \nocite{*}
    \printbibliography[heading=none]
\end{frame}

\end{document}


% =============================================================== %
% More example slides, derived from https://www.overleaf.com/latex/templates/metropolis-beamer-theme/qzyvdhrntfmr %
% =============================================================== %

% \begin{frame}{Blocks}
%   Three different block environments are pre-defined and may be styled with an optional background color.
%   \metroset{block=fill}
%   \begin{block}{Default}
%     Block content.
%   \end{block}
%   \begin{alertblock}{Alert}
%     Block content.
%   \end{alertblock}
%   \begin{exampleblock}{Example}
%     Block content.
%   \end{exampleblock}
% \end{frame}

% \begin{frame}[fragile]{Typography}
%       \begin{verbatim}The theme provides sensible defaults to
% \emph{emphasize} text, \alert{accent} parts
% or show \textbf{bold} results.\end{verbatim}

%   \begin{center}becomes\end{center}

%   The theme provides sensible defaults to \emph{emphasize} text,
%   \alert{accent} parts or show \textbf{bold} results.
% \end{frame}

% \begin{frame}{Animation}
%   \begin{itemize}[<+- | alert@+>]
%     \item \alert<4>{This is\only<4>{ really} important}
%     \item Now this
%     \item And now this
%   \end{itemize}
% \end{frame}

% \begin{frame}{Font feature test}
%   \begin{itemize}
%     \item Regular
%     \item \textit{Italic}
%     \item \textsc{SmallCaps}
%     \item \textbf{Bold}
%     \item \textbf{\textit{Bold Italic}}
%     \item \textbf{\textsc{Bold SmallCaps}}
%     \item \texttt{Monospace}
%     \item \texttt{\textit{Monospace Italic}}
%     \item \texttt{\textbf{Monospace Bold}}
%     \item \texttt{\textbf{\textit{Monospace Bold Italic}}}
%   \end{itemize}
% \end{frame}

% \begin{frame}{Figures}
%   \begin{figure}
%     \newcounter{density}
%     \setcounter{density}{20}
%     \begin{tikzpicture}
%       \def\couleur{alerted text.fg}
%       \path[coordinate] (0,0)  coordinate(A)
%                   ++( 90:5cm) coordinate(B)
%                   ++(0:5cm) coordinate(C)
%                   ++(-90:5cm) coordinate(D);
%       \draw[fill=\couleur!\thedensity] (A) -- (B) -- (C) --(D) -- cycle;
%       \foreach \x in {1,...,40}{%
%           \pgfmathsetcounter{density}{\thedensity+20}
%           \setcounter{density}{\thedensity}
%           \path[coordinate] coordinate(X) at (A){};
%           \path[coordinate] (A) -- (B) coordinate[pos=.10](A)
%                               -- (C) coordinate[pos=.10](B)
%                               -- (D) coordinate[pos=.10](C)
%                               -- (X) coordinate[pos=.10](D);
%           \draw[fill=\couleur!\thedensity] (A)--(B)--(C)-- (D) -- cycle;
%       }
%     \end{tikzpicture}
%     \caption{Rotated square from
%     \href{http://www.texample.net/tikz/examples/rotated-polygons/}{texample.net}.}
%   \end{figure}
% \end{frame}
% \begin{frame}{Tables}
%   \begin{table}
%     \caption{Largest cities in the world (source: Wikipedia)}
%     \begin{tabular}{lr}
%       \toprule
%       City & Population\\
%       \midrule
%       Mexico City & 20,116,842\\
%       Shanghai & 19,210,000\\
%       Peking & 15,796,450\\
%       Istanbul & 14,160,467\\
%       \bottomrule
%     \end{tabular}
%   \end{table}
% \end{frame}

% \begin{frame}{Math}
%   \begin{equation*}
%     e = \lim_{n\to \infty} \left(1 + \frac{1}{n}\right)^n
%   \end{equation*}
% \end{frame}

% \begin{frame}{Line plots}
%   \begin{figure}
%     \begin{tikzpicture}
%       \begin{axis}[
%         mlineplot,
%         width=0.9\textwidth,
%         height=6cm,
%       ]

%         \addplot {sin(deg(x))};
%         \addplot+[samples=100] {sin(deg(2*x))};

%       \end{axis}
%     \end{tikzpicture}
%   \end{figure}
% \end{frame}

% \begin{frame}{Bar charts}
%   \begin{figure}
%     \begin{tikzpicture}
%       \begin{axis}[
%         mbarplot,
%         xlabel={Foo},
%         ylabel={Bar},
%         width=0.9\textwidth,
%         height=6cm,
%       ]

%       \addplot plot coordinates {(1, 20) (2, 25) (3, 22.4) (4, 12.4)};
%       \addplot plot coordinates {(1, 18) (2, 24) (3, 23.5) (4, 13.2)};
%       \addplot plot coordinates {(1, 10) (2, 19) (3, 25) (4, 15.2)};

%       \legend{lorem, ipsum, dolor}

%       \end{axis}
%     \end{tikzpicture}
%   \end{figure}
% \end{frame}

% {%
% \setbeamertemplate{frame footer}{My custom footer}
% \begin{frame}[fragile]{Frame footer}
%     Metropolis defines a custom beamer template to add a text to the footer. It can be set via
%     \begin{verbatim}\setbeamertemplate{frame footer}{My custom footer}\end{verbatim}
% \end{frame}
% }

% \begin{frame}{References}
%   Some references to showcase [allowframebreaks] \cite{knuth92,ConcreteMath,Simpson,Er01,greenwade93}
% \end{frame}

% \begin{frame}[fragile]{Backup slides}
%   Sometimes, it is useful to add slides at the end of your presentation to
%   refer to during audience questions.

%   The best way to do this is to include the \verb|appendixnumberbeamer|
%   package in your preamble and call \verb|\appendix| before your backup slides.

%   Metropolis will automatically turn off slide numbering and progress bars for
%   slides in the appendix.
% \end{frame}
